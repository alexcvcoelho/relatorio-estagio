%% abtex2-modelo-trabalho-academico.tex, v-1.9.6 laurocesar
%% Copyright 2012-2016 by abnTeX2 group at http://www.abntex.net.br/ 
%%
%% This work may be distributed and/or modified under the
%% conditions of the LaTeX Project Public License, either version 1.3
%% of this license or (at your option) any later version.
%% The latest version of this license is in
%%   http://www.latex-project.org/lppl.txt
%% and version 1.3 or later is part of all distributions of LaTeX
%% version 2005/12/01 or later.
%%
%% This work has the LPPL maintenance status `maintained'.
%% 
%% The Current Maintainer of this work is the abnTeX2 team, led
%% by Lauro César Araujo. Further information are available on 
%% http://www.abntex.net.br/
%%
%% This work consists of the files abntex2-modelo-trabalho-academico.tex,
%% abntex2-modelo-include-comandos and abntex2-modelo-references.bib
%%

% ------------------------------------------------------------------------
% ------------------------------------------------------------------------
% abnTeX2: Modelo de Trabalho Academico (tese de doutorado, dissertacao de
% mestrado e trabalhos monograficos em geral) em conformidade com 
% ABNT NBR 14724:2011: Informacao e documentacao - Trabalhos academicos -
% Apresentacao
% ------------------------------------------------------------------------
% ------------------------------------------------------------------------

\documentclass[
	% -- opções da classe memoir --
	12pt,				% tamanho da fonte
	%openright,			% capítulos começam em pág ímpar (insere página vazia caso preciso)
	%twoside,			% para impressão em recto e verso. Oposto a oneside
	oneside,			% impressão de um lado só
	a4paper,			% tamanho do papel. 
	% -- opções da classe abntex2 --
	chapter=TITLE,		% títulos de capítulos convertidos em letras maiúsculas
	section=TITLE,		% títulos de seções convertidos em letras maiúsculas
	%subsection=TITLE,	% títulos de subseções convertidos em letras maiúsculas
	%subsubsection=TITLE,% títulos de subsubseções convertidos em letras maiúsculas
	sumario=tradicional % opção de sumário
	% -- opções do pacote babel --
	english,			% idioma adicional para hifenização
	french,				% idioma adicional para hifenização
	spanish,			% idioma adicional para hifenização
	brazil				% o último idioma é o principal do documento
	]{abntex2}

% ---
% Pacotes básicos 
% ---
\usepackage{lmodern}			% Usa a fonte Latin Modern			
\usepackage[T1]{fontenc}		% Selecao de codigos de fonte.
\usepackage[utf8]{inputenc}		% Codificacao do documento (conversão automática dos acentos)
\usepackage{lastpage}			% Usado pela Ficha catalográfica
\usepackage{indentfirst}		% Indenta o primeiro parágrafo de cada seção.
\usepackage{color}				% Controle das cores
\usepackage{graphicx}			% Inclusão de gráficos
\usepackage{microtype} 			% Melhorias de justificação
\usepackage{hyperref}
\usepackage{facens}				% Padrão Facens
\usepackage{pdfpages}			% Include de pdfs
\usepackage{lipsum}				% Geração de dummy text
\usepackage[alf]{abntex2cite}	% Citações padrão ABNT
%\usepackage[brazilian,hyperpageref]{backref}	 % Paginas com as citações na bibl
% ---

% ---
% Indicando pasta de figuras\\
% ---
\graphicspath{{imagens/}}
% ---

% --- 
% CONFIGURAÇÕES DE PACOTES
% --- 

% ---
% Configurações do pacote backref
% Usado sem a opção hyperpageref de backref
%\renewcommand{\backrefpagesname}{Citado na(s) página(s):~}
% Texto padrão antes do número das páginas
%\renewcommand{\backref}{}
% Define os textos da citação
%\renewcommand*{\backrefalt}[4]{
%	\ifcase #1 %
%		Nenhuma citação no texto.%
%	\or
%		Citado na página #2.%
%	\else
%		Citado #1 vezes nas páginas #2.%
%	\fi}%
% ---

% ---
% Informações de dados para CAPA e FOLHA DE ROSTO
% ---
\titulo{RELATÓRIO DE ESTÁGIO SUPERVISIONADO OBRIGATÓRIO}
\autor{Alex Covolan Vieira Coelho}
\local{Sorocaba/SP}
\data{2017}
\orientador[Dra.]{Andrea Vieira Braga}
\instituicao{Faculdade de Engenharia de Sorocaba - FACENS}
\tipotrabalho{Dissertação}

% O preambulo deve conter o tipo do trabalho, o objetivo, 
% o nome da instituição e a área de concentração 
\preambulo{Relatório apresentado como requisito obrigatório para a integralização do Curso de Engenharia da Computação.}
% ---


% ---
% Configurações de aparência do PDF final

% alterando o aspecto da cor azul
\definecolor{blue}{RGB}{41,5,195}

% informações do PDF
\makeatletter
\hypersetup{
     	%pagebackref=true,
		pdftitle={\@title}, 
		pdfauthor={\@author},
    	pdfsubject={\imprimirpreambulo},
	    pdfcreator={LaTeX with abnTeX2},
		pdfkeywords={abnt}{latex}{abntex}{abntex2}{trabalho acadêmico}, 
		colorlinks=false,       		% false: boxed links; true: colored links
    	linkcolor=blue,          	% color of internal links
    	citecolor=blue,        		% color of links to bibliography
    	filecolor=magenta,      		% color of file links
		urlcolor=blue,
		bookmarksdepth=4
}
\makeatother
% --- 

% --- 
% Espaçamentos entre linhas e parágrafos 
% --- 

% O tamanho do parágrafo é dado por:
\setlength{\parindent}{1.25cm}

% Controle do espaçamento entre um parágrafo e outro:
\setlength{\parskip}{0.2cm}  % tente também \onelineskip

% ---
% compila o indice
% ---
\makeindex
% ---

% ----
% Início do documento
% ----
\begin{document}
% Seleciona o idioma do documento (conforme pacotes do babel)
%\selectlanguage{english}
\selectlanguage{brazil}

% Retira espaço extra obsoleto entre as frases.
\frenchspacing 

% ----------------------------------------------------------
% ELEMENTOS PRÉ-TEXTUAIS
% ----------------------------------------------------------
\pretextual

%\includepdf{report4.pdf}


% ---
% Capa
% ---
\imprimircapa
% ---

% ---
% Folha de rosto
% (o * indica que haverá a ficha bibliográfica)
% ---
\imprimirfolhaderosto*
% ---

% ---
% Inserir a ficha bibliografica
% ---

% Isto é um exemplo de Ficha Catalográfica, ou ``Dados internacionais de
% catalogação-na-publicação''. Você pode utilizar este modelo como referência. 
% Porém, provavelmente a biblioteca da sua universidade lhe fornecerá um PDF
% com a ficha catalográfica definitiva após a defesa do trabalho. Quando estiver
% com o documento, salve-o como PDF no diretório do seu projeto e substitua todo
% o conteúdo de implementação deste arquivo pelo comando abaixo:
%
% \begin{fichacatalografica}
%     \includepdf{fig_ficha_catalografica.pdf}
% \end{fichacatalografica}

%\include{pre-textuais/fichacatalografica}
% ---

% ---
% Inserir errata
% ---
%\include{pre-textuais/errata}
% ---

% ---
% Inserir folha de aprovação
% ---

% Isto é um exemplo de Folha de aprovação, elemento obrigatório da NBR
% 14724/2011 (seção 4.2.1.3). Você pode utilizar este modelo até a aprovação
% do trabalho. Após isso, substitua todo o conteúdo deste arquivo por uma
% imagem da página assinada pela banca com o comando abaixo:
%
% \includepdf{folhadeaprovacao_final.pdf}
%
%\include{pre-textuais/folhaaprovacao}
% ---

% ---
% Dedicatória
% ---
%\include{pre-textuais/dedicatoria}
% ---

% ---
% Agradecimentos
% ---
%\include{pre-textuais/agradecimentos}
% ---

% ---
% Epígrafe
% ---
%\include{pre-textuais/epigrafe}
% ---

% ---
% RESUMOS
% ---

% resumo em português
%\include{pre-textuais/resumo/pt_Br}

% resumo em inglês
%\include{pre-textuais/resumo/en}

% ---

% ---
% inserir lista de ilustrações
% ---
\include{pre-textuais/lista-ilustracoes}
% ---

% ---
% inserir lista de tabelas
% ---
%\include{pre-textuais/lista-tabelas}
% ---

% ---
% inserir lista de abreviaturas e siglas
% ---
%\begin{siglas}
  \item[CRTSE] Centro Regional de Tecnologia Santa Escolástica
  \item[CRTS] Companhia Rede Telefônica Sorocabana
  \item[ACRTS] Associação Cultural de Renovação Tecnológica Sorocabana
  \item[IPEAS] Instituto de Pesquisas e Estudos Avançados Sorocabano
  \item[LEMAT] Laboratório de Ensaio de Materiais
  \item[ABMES] Associação Brasileira de Mantenedoras de Ensino Superior
\end{siglas}
% ---

% ---
% inserir lista de símbolos
% ---
%\include{pre-textuais/lista-simbolos}
% ---

% ---
% inserir o sumario
% ---
\include{pre-textuais/sumario}
% ---



% ----------------------------------------------------------
% ELEMENTOS TEXTUAIS
% ----------------------------------------------------------
\textual
% ---
\pagestyle{simple}


\chapter{Introdução}
\label{chap:cap1}


\chapter{Plano de Estágio}
\label{chap:cap2}

\section{Identificação do Aluno}
\label{sec:identaluno}
\textbf{Nome:} Alex Covolan Vieira Coelho\\
\indent \textbf{Matrícula: } 132115\\
\indent \textbf{Curso: } Engenharia da Computação\\
\indent \textbf{Semestre: } 10º\\
\indent \textbf{Ano de ingresso: } 2013\\
\indent \textbf{E-mail: } alexcvcoelho@gmail.com\\

\section{Empresa}
\label{sec:empresa}

\section{Estágio}
\label{sec:estagio}

\section{Supervisor de Estágio na Empresa}
\label{sec:supestagioempre}

\section{Atividades Programadas Para o Estagiário}
\label{sec:ativestagiario}

\chapter{Organograma da Empresa}
\label{chap:chap3}

\section{A Empresa}
\label{sec:aempresa}

\section{Objeto de Produção da Empresa e Missão}
\label{sec:prodmissaoempresa}

\section{Organograma do Setor}
\label{sec:aempresa}

\section{Atribuições do Setor}
\label{sec:atribsetor}

\section{Processo de Seleção}
\label{sec:procselecao}

\chapter{Recursos disponíveis}
\label{chap:chap4}

\section{Oficinas e Laboratórios}
\label{sec:oficlabs}

\section{Equipe de Trabalho}
\label{sec:equipetrabalho}

\section{Inter-relação com Outras Áreas da Empresa}
\label{sec:relacaoareas}

\chapter{Atividades Desenvolvidas}
\label{chap:chap5}

\section{Áreas de Identificação com o Curso}
\label{sec:identcurso}

\chapter{Conclusões}
\label{chap:chap6}

% ----------------------------------------------------------
% Finaliza a parte no bookmark do PDF
% para que se inicie o bookmark na raiz
% e adiciona espaço de parte no Sumário
% ----------------------------------------------------------
\phantompart

% ---
% Conclusão
% ---
%\chapter{Conclusão}
\label{chap:conclusao}
Concluir sobre o trabalho, apresentar pontos de dificuldades de sua aplicação, pontos a serem melhorados e trabalhos futuros

% ----------------------------------------------------------
% ELEMENTOS PÓS-TEXTUAIS
% ----------------------------------------------------------
\postextual
% ----------------------------------------------------------

% ----------------------------------------------------------
% Referências bibliográficas
% ----------------------------------------------------------
%\bibliography{pos-textuais/bibliografia}


%---------------------------------------------------------------------
% INDICE REMISSIVO
%---------------------------------------------------------------------
%\phantompart
\printindex
%---------------------------------------------------------------------

\end{document}
