\chapter{Especificações e Arquitetura do Sistema}
\label{chap:arquitetura}
Apresentar o diagrama de caso de uso do sistema, juntamente com a descrição do funcionamento do sistema como um todo, mostrando os requisitos dos usuários que serão afetados por esta tecnologia.

Apresentação do esquema de funcionamento do sistema, mostrando aonde estará empregada cada tecnologia utilizada no sistema. Juntamente será apresentado os tipos de dados trafegados pelas interfaces presentes, mostrando uma arquitetura distribuída e a integração das partes desta.

\section{Tecnologias}
\label{sec:tecnologias}
Mostrar de forma específica cada tecnologia que será empregada

\subsection{Clojure}
Apresentar a linguagem de programação \textit{Clojure} e como ela é empregada no trabalho.

\subsection{Docker e Docker Swarm}
Apresentar o uso de containers \textit{Docker} para encapsular componentes do sistema e \textit{Docker Swarm} para balancear cargas entre containers.

\subsection{Redis}
Mostrar o uso de um \textit{cache} dentro do sistema, sua importância e melhora no desempenho do sistema.

\subsection{Kafka}
Apresentar seu uso para realizar funções em tempo real dentro do sistema, bem como seu funcionamento.

\subsection{MongoDB}
Apresentar as características do banco NoSQL seu uso e diferenças entre um banco SQL, bem como seus pontos positivos.

\subsection{Datomic}
Apresentar uma nova forma de armazenar dados, se utilizando de histórico, sempre mantendo o estado anterior dos dados.

